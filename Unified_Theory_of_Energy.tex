\documentclass[12pt]{article}
\usepackage{amsthm}

\begin{document}
\title{The Unified Theory of Energy}
\author{Michael Vera}
\date{2019}
\maketitle

\begin{abstract}
If a more comprehensive definition of energy can be determined, if the logic behind the functioning of energy can be improved, the result would be a truly generalized theory of energy which would lead to improved formulae for the estimation, calculation, and even definition of potential energy.
\end{abstract}


\section*{Introduction}
The present state of electrical science seems peculiarly favorable to speculation. The concept of the distribution of electricity on the surface of conductors has led to an excellent set of equations for estimating electron motion through a closed loop. 

These equations have served for nearly five generations now, but when these estimations are quickly inducted as Laws, and the idea is propogated that there could be no further room for the advancement of the concept of energy, and we find ourselves with dozens of seemingly unsolvable riddles that arise from our varied and flawed courses of logic and language, and we consume more and more geological resources to fuel our need for electron motion using theories based on tubes and steam engines and giant, billowing machines, it may be time to move on to a less step-motherly understanding of energy.



\newpage
\section{Energy}

\subsection{Energy State Theorum}
\newtheorem{thm}{Theorem}


\begin{thm}
   Energy exists in three distinct states: as Radiation, as Particulate Motion, and as Gravitation. Each of the three energy states cannot exist apart from, or without, the other states.
\end{thm}

\theoremstyle{definition}
\newtheorem{defn}{Definition}

\begin{defn}
   Radiation is Energy extended outwardly, intended for absorption.
\end{defn}

\begin{defn}
   Gravitation is Energy stored within Mass; it is absorbed Radiation and it is Potential Energy.
\end{defn}

\begin{defn}
   Particulate Motion is inertial Energy affecting Mass and Gravitation while being affected by Radiation and Mass.
\end{defn}

\begin{defn}
   A Mass Structure is a collection of Particles drawn together by Gravitation while held apart by Radiation and Particulate Motion.
\end{defn}

\subsection{Energy Storage Theorum}
\begin{thm}
   Radiation is absorbed and stored as Gravitation inside a Mass Structure.
\end{thm}

Radiation would be instantaneous and infinite, but is slowed and limited by Particulate Motion which must interact with the Radiation.


\subsection{Energy Emission Theorum}
\begin{thm}
   Excess Gravitation within a Mass Structure is shed as Radiation.
\end{thm}

\begin{defn}
   Anything emitting Radiation is a Radiation Source.
\end{defn}

\begin{thm}
   Each Radiation Source will have its own Coordinate System dictated by the total number, and therefore the total possible motion, of Particles within that Radiation Source.
\end{thm}

The number of Particles dictates how far Radiation can extend within the Radiation Coordinate System. 

Every Radiation Coordinate System will be completely contained within a larger Radiation Coordinate System. An Electron will be contained within the reach of its Nucleus. An Atom will be contained within a Molecule. A Planet will be contained within a Solar System. A Star will be contained within a Galaxy. A Galaxy will be contained within a Universe. A Universe will be contained within an Electron. 

From this we can be sure a cloud of Particles orbits an Electron, which is also a Galaxy on some Scale.

\begin{defn}
   Scale refers to the relative size of any Radiation Source.
\end{defn}

Every Radiation Source will have an "Electron" within its Scale at the appropriate level. The Electron itself can be a Radiation Source being orbited by smaller Particles like "planets" and those again being orbited by smaller Particles like "moons" until the smallest functional Particle below that specific Radiation Source is found, which would then also be called an Electron.

We can, thus, use Scale to consider the Sun as a Radiation Source being orbited by eight Particles and realize our Solar System is an Oxygen Molecule on a different Scale. This would also prove our Galaxy to be an Aether with respect to Scale. This further proves that, on some specific Scale, our own atmosphere's breathable air is also vacuuous.

\begin{thm}
   Due to the interaction of Radiation with Particulate Motion, the Frequency, or Frequencies, of Radiation is limited to that of the surface of the Mass Structure from which the Radiation is emitted.
\end{thm}

Without interaction with Particles, Radiation emitted by a Mass Structure would be instantaneous with respect to time or velocity or other such constructs. But since Radiation emitted from a Mass Structure cannot exist without Particle interaction, it is then limited to the specific set of Frequencies of the Mass Structure.

\begin{thm}
   Radiation cannot have a Frequency of zero, and will contain a subset of all possible Frequencies.
\end{thm}

\begin{defn}
   God is the one Radiation Source contanining the set of all possible Frequencies.
\end{defn}

All of this indicates interactions occur on the Surface of all Mass Structures as Radiation is stored and shed.



\newpage
\section*{Mass}

\begin{thm}
   Mass is made up of Particles of infinitely diminishing size which group together and break apart; which can be attracted to, or repelled by, other Particles.
\end{thm}

Only Particles have Mass.  All Particles have Mass. All Particles are Mass.

Only Particles can store Gravitation.

Particles absorb Radiation which is stored as Gravitation.

On some Scale, Particles tend to form spheres which rotate about an Axis.


\begin{defn}
   A Mass Structure is a specific arrangement of Particles tied together by Gravitation while held apart by Particulate Motion and Radiation.
\end{defn}

\subsection{Particle State Theorem}
\begin{thm}
   Particles have three states: Undergravitated, Overgravitated, and Even.
\end{thm}

Particulate state also depends on Scale. An Overgravitated particle could appear to be Undergravitated on a different Scale, such as when clustered in a group of like particles.

Particles with excess Gravitation emit Radiation, repelling other Overgravitated Particles, while attracting Undergravitated Particles.

A Particle in orbit around a certain Radiation Source is Even with respect to the Scale of that Radiation Source, and also Overgravitated with respect to Particles smaller than itself. Further, it might appear Undergravitated with respect to the Radiation Source completely containing the Radiation Source that Particle is orbiting.

Particles can exit the Mass Structure of a Radiation Source at its equator, pressed outward by the emitted Radiation or Overgravitation. 

Particles can return to a Radiation Source's Mass Structure at its poles, if not intercepted on the way.







\newpage
\section*{Surface Interactions}

The interception of Radiation by a Mass Structure rarely lends itself to the perfect storage of all Radiation directly into Gravitation. And while Particles trapped within the Mass Structure may be relatively stable in their orbits, those at or near the surface of the Mass Structure are most succeptible to various influences. An orbiting Particle excited to a near free state which is then influenced by a strong pulse of Radiation at the correct Frequency could force that Particle out of its orbit; especially if, for example, an Atom already suffered a spare Electron in its outer valence.

\begin{defn}
   A Surface Interaction describes the remainder of the process of the storage or extraction of Energy.
\end{defn}

The most simple Surface Interaction might be the absorption of Radiation into a Mass Structure that looses a single Electron, which rejoins another similar Atom in the surface of the Mass Structure that has a valence one Electron short. It should also be possible for a star to spit out a Particle that might extend a very long way before returning to a pole. It also has the chance to be attached to any other Mass Structure it may encounter on the trip.

Surface Interactions increase in complexity by degrees. Each new degree is defined by its utilization of each previous degree.

\begin{thm}
   A First Degree Surface Interaction is any transfer of Energy that leaves, as a remainder, one or more Particles removed from their Radiation Source.
\end{thm}

A more complex example of a First Degree Surface Interaction might be a Particle which was Radiated directly away from its Mass Structure then becoming subject to the next strongest Radiation Source. If the next strongest Radiation Source were a nearby star, the Particle would travel antiparallel to that star's Radiation, and be deposited at its nearest pole, joining the surface of the star's Mass Structure. 

It would also be possible for that Particle to balance perfectly between the poles of the star, unable to return to the Mass Structure and held at orbit by the Radiation emanating from the star; like a ball on the top of a column of air or water. The distance of its orbit is, of course, determined by the relative size of the particle and the total particles within that Radiation Coordinate System.


\begin{thm}
   A Second Degree Surface Interaction is any transfer of Energy whose First Degree Interaction remainder Particles interact with the Mass Structure to form a new type of Atomic Structure at the surface.
\end{thm}

\begin{thm}
   An Atomic Structure is a subset of its overall Mass Structure.
\end{thm}

More likely than a Particle breaking free of its Mass Structure and becoming subject to a new Radiation Source is that same Particle smashing into its own Mass Structure. This can physically break loose other Particles at the surface of the Mass Structure, forming new Atomic Structures. These Atomic Structures will generally be gasses which are generated by the interaction of Radiation of a specific set of Frequencies with a Mass Structure which has Particulate Motion of a specific set of Frequencies. Certain Mass and Radiation combinations will produce higher likelihood of specific Atomic Structures being generated.

A simple example of a Second Degree Surface Interaction might be a Free Electron  knocking loose another particle which has just enough total Gravitation to attract the Free Electron into an Atomic Structure of Hydrogen. If the Mass Structure from which the newly formed Hydrogen contains enough Gravitation, it will keep the Hydrogen at the surface of the Mass Structure. If not, the Hydrogen will succumb to the greatest Radiation Source, possibly of a larger Scale.

The complexity of the newly formed Atomic Structure is dependent on probabilities based on the complexity of any incoming Particles, the intensity and frequency of the Radiation and Gravitation, and a total number of Particles within that Radiation Coordinate System.

As Second Degree Surface Interactions develop, Hydrogen could be generated along with the occassional Oxygen, based on their given probabilities, and water could easily form. Salts, minerals, acids, oxides, and more complex molecules can form, based on their probabilities. Each of these will interact with Radiation in new ways.

\begin{defn}
   A Generator is a Mass Structure which interacts with a Radiation Source to create new Atomic Structures of a specific arrangement.
\end{defn}

A large amount of Iron interacting with Radiation from a yellow dwarf star is a Water Generator utilizing the Second Degree of Surface Interaction.

Along with water, acids, bases and salts are also the result of Second Degree Surface Interactions which develop depending on the probabilities of the various Generators that form, which themselves are dependent on the specific subset of Frequencies and Particles available from its First Degree Surface Interactions.


\begin{thm}
   A Third Degree Surface Interaction results in a phyisical change to the Mass Structure.
\end{thm}

Third Degree Surface Interactions will include multiple Second Degree Generators of varying types. Simple gasses will form liquids, and more complex compounds. Proteins and nucleic acids form and further chemical reactions start to occur as the compounds continue to interact with Radiation. Chlorophyl is developed, even hormones appear. 

There will be enough Gravitation by the advent of Third Degree of Surface Interactions that any gasses generated will likely stay attached or close to the Mass Structure.

It should be possible to find a Mass Structure that is able to store more Gravitation than its Mass Structure can handle. In this case, the Mass Structure is expanded by the increased Gravitation and the proportional increase in its Particles' orbital radii. If the Mass is unable to maintain its structure, it will split, break, or crumble, and quickly release its excess Gravitation. If an Atom, it may split into two Atoms, each of smaller configuration. If a star, it will likely expel some amount of Mass along with emission of its excess Gravitation. This could form a future satellite or planet or it may fall back into the star and rejoin its Mass Structure.

Some Mass Structures will give up Mass along with Radiation emission only to restructure it at the poles, allowing an even greater capacity for Gravitation to be stored within the Mass Structure. Perhaps as the internal Gravitation increases, the Mass Structure expands to force internal Mass to the surface, creating mountains, continuing the example of an iron planet. A magnet will constantly shed spare electrons, and even small bits of iron oxide only to have them restructured at its poles.

\begin{thm}
   A Mass Structure will grow to form a spherical shape, due to the stacking of Particles, and will expel excess Gravitation as Radiation outwardly on a plane at or near its equator with higher entropy, and reclaim Radiation and Particles at its poles with lower entropy.
\end{thm}


\begin{defn}
   Surface Depth is the depth of saturation of Radiation into a Mass Structure.
\end{defn}

\begin{thm}
  Surface Depth is proportional to the amount of Gravitation that a Mass Structure is able to hold.
\end{thm}





\newpage
\section*{Advanced Surface Interactions}

\begin{defn}
   Life is any system which utilizes the results of Surface Interactions to separate itself from the Surface of a Mass Structure.
\end{defn}

\begin{defn}
  A Fourth Degree Surface Interaction includes the beginning of Life.
\end{defn}

Unicellular Organisms are the result of Fourth Degree Surface Interactions and include bacteria, archaea, protozoa, algae, fungi, and viruses. They utilize nearby protiens and acids generated in the Third Degree of Surface Interactions to form cell walls, thereby separating and protecting themselves from the Surface. They also form cells that find more specialized ways to interact with Radiation and Gravitation, such as chloroplasts. Energy storage molecules are also developed in Fourth Degree Surface Interactions. Most Fourth Degree Surface Interactions are developed completely within a liquid.

These organisms can develop internal Second Degree Generators to make resources more directly available, or to have control over where the resources are made. They can also become mobile, learning how to hunt for more resources than were available in a stationary position.


\begin{defn}
   A Fifth Degree Surface Interaction is a collection or grouping of various types of specialized cells that work together to further remove itself from the Surface.
\end{defn}


By the Fifth Degree of Surface Interactions, cells have various advanced processes available to them; transport, adhesion, movement, signalling, repair, metabolism, splicing, and even autophagy are available to cells which can allow for more complex organisms to develop. Most cell development has been within liquid to this point, so those liquid-bound organisms are likely to be the most advanced; for example fish might be quite developed while plants are just begining to break free of the liquid. Plants are certainly another result of the Fifth Degree of Surface Interactions and are cruicial for there to be a Sixth Degree of Surface Interactions. While plants certainly found root in their liquid environment, they are the first lifeform to truly separate from the liquid Surface. Plants require enough of the Second Degree of Surface Interactions for a somewhat solid surface to form, and enough of the Third Degree of Surface Interactions for a transitional material with which to escape the liquid.

Highly specialized cell processes begin to form that allow organisms to sense specific Frequencies of Radiation. Some are meant for Frequencies in the Infrared spectrum through touch and feel. Some are meant for other subsets of Frequencies such as for vision or hearing. This is done through advanced storage of those Radiation Frequencies within highly specialized cells. A cross-section of the extended Radiation of a distant star can be captured and held within cells attached to complex systems that are meant to store Radiation. This indicates Gravitation is essential to the storage of Radiation, which might be called Memory.

\begin{defn}
   Memory is the storage of a subset of all Frequencies of Radiation as Gravitation within highly specialized cells which can also be retreived in a reverse process. 
\end{defn}

It is likely that the retreival of Gravitation from these cells is not perfectly efficient and, as such, some amount of that subset of Frequencies of Radiation stored as Graviation cannot be retreived, providing both retrievable and permanent storage in the same cell.

\begin{thm}
   Radiation tends to be stored in Gravitation as a specific Frequency or set of Frequencies.
\end{thm}

\begin{thm}
   Radiation tends to be retreived from its Gravitated state at the same Frequency or Frequencies in which it was stored.
\end{thm}

This may depend on the total entropy and any change in the total Particles of the specific Radiation Coordinate System.


\begin{defn}
   The Sixth Degree Surface Interaction are the most complex organisms, which are a grouping of highly specialized groupings of cells with specialized processes available to make the best use of the results of each of the five previous Surface Interactions.
\end{defn}

While still dependent on each of the previous Degrees of Surface Interaction, organisms of the Sixth Degree of Surface Interactions are fully mobile and separate from the Surface. They are also the first organisms to understand their position on the Surface and each previous position, whereas each previous Degree of Surface Interaction is not self-aware. Liquid-bound organisms should have some of the most advanced examples of organisms to this point; for example whales and dolphin should display signs of self-awareness, measurable as Anxiety. 

\begin{thm}
   Anxiety is a measure of self-awareness in an organism in the Sixth Degree of Surface Interactions.
\end{thm}






\newpage
\section*{Synthetic Constructs}

\begin{defn}
   A Synthetic Construct is an idea or concept generated as a result of very advanced Surface Interactions to assist with perception of the Surface Interactions themselves.
\end{defn}


\subsection{The Infinite Straight Line}

This may be the simplest, and worst, Synthetic Construct we have. Clearly everything attempts to become a spherical object and is made of spherical objects on every scale, and usually exists within or on spherical objects, while spinning in circles and travelling in ellipses around larger spherical objects, proves nothing is straight, and that even the most perfect ruler, when scaled correctly, will most certainly not be "straight". 

And even if something can give the illusion of being "straight", it will certainly not stay straight for all eternity. An Electron cannot maintain "straight" for any more than the length of one radian of the orbit it is exiting tangentially. And would you allow your life to depend on a straight line the length of one radian of an Electron's orbit? Yet clearly man can develop a ruler accurate to a fraction of a millimeter; which contains at its edge a very very fuzzy array of elliptical Particle Motion. But even with the most accurate ruler, if you have the time to line them up end to end in a straight line, will they really be perfectly straight for all eternity? Will they map the motion or existence of anything at all?



\subsection{Cartesean Coordinate System}

Can anything good come from the Synthetic Construct of the Infinite Straight Line? The answer is, of course, no. Yet there it is, in all its glory. A coordinate system made from a whole quiver full of Infinitely Straight Lines arranged at perfect Right Angles going off forever somewhere and then forever some more. And when that ends, it keeps going straight. 

The Cartesian Coordinate System is, in fact, a subset, reduction, and simplification of the Radiation Coordinate System; if you were to chop off anything that didn't appear to be at right angles.

It is hoped that by now you realize the Cartesean Coordinate System is the "training wheels" of physics. Its "kindergarten math" for people with stopwatches and thermometers who think everything must be solved for time or temperature. Who think that everything exists only on a perfectly flat surface at the one Scale that is most easily observed with the senses. 


\subsection{Time}

The Synthetic Construct of Time is an attempt to mechanically maintain a count of whole numbers with the false assumption that Time is a constant following an Infinite Straight Line. Time, unfortunately, Scales along with the other Synthetic Constructs to mean little when Scaled to any other size than for small creatures on the surface of a large Mass Structure. The Electron certainly doesn't care how many Earth-spins it took for Water to form on the surface. Maybe the Electron is only concerned about what its going to do 365.2425 Electron-spins from now after it ellipses its Nucleus once and when its taxes will again be due.


\subsection{Temperature}

The Synthetic Construct of Temperature is highly subjective and suffers the worst Scaling issues, preventing Temperature from being useful for anything other than determining Human comfort. It may be more useful to consider "Temperature" as the results of a Radiation sensor measurement within the "Infrared" Frequency subset. This is certainly not something to ever solve partial differential equations for, even less so than Time.



\end{document}


